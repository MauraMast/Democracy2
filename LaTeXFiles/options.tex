%\usepackage[polutonikogreek,english]{babel}

%\usepackage{dingbat}
\usepackage{amsfonts}
\usepackage{amsmath}
\usepackage{amssymb}
\usepackage{amsthm}
\usepackage{textcomp}
\usepackage[margin=1in, paperwidth=8.5in, paperheight=11in]{geometry}
%\usepackage{ahmacros}
\usepackage{tikz}
\usetikzlibrary{calc}
\usetikzlibrary{intersections}
\usetikzlibrary{decorations.pathmorphing}
\usetikzlibrary{decorations.pathreplacing}
\usetikzlibrary{shapes.geometric}
\usetikzlibrary{fadings}
\usetikzlibrary{patterns.meta}
\usepackage{array}
\usepackage{multicol}
\usepackage{longtable}
\usepackage{array}
\usepackage{ragged2e}
\usepackage[pdfpagelabels]{hyperref}                 % For creating hyperlinks in cross references
\usepackage{calc}
\usepackage{fullpage}
%\usepackage{enumerate}
\usepackage{graphicx}
\usepackage{makeidx}
\usepackage{wrapfig}
%\usepackage{geometry}                % See geometry.pdf to learn the layout options. There are lots.
%\geometry{letterpaper}                   % ... or a4paper or a5paper or ... 
\usepackage{fancyhdr}
%\usepackage{multicols}
\usepackage{enumitem}
\usepackage{pgffor} % for \pgfmathsetmacro and \breakforeach
\usepackage{listofitems} % to create arrays with \readlist
\usepackage{ifthen} % for \whiledo

\pagestyle{fancy}
\setlength{\headheight}{30pt}


%\parindent 1cm
%\parskip 0.2cm
%\topmargin 0.2cm
%\oddsidemargin 1cm
%\evensidemargin 0.5cm
%\textwidth 15cm
%\textheight 21cm
%\parindent 1cm
%\parskip 0.2cm
%\topmargin -0.25in
%\headheight 0.25in
%\headsep 0.25in
%\oddsidemargin 0.5in
%\evensidemargin 0in
%\textwidth 6in
%\textheight 8.5in
\newtheorem{theorem}{Theorem}[section]
\newtheorem{proposition}[theorem]{Proposition}
\newtheorem{corollary}[theorem]{Corollary}
\newtheorem{lemma}[theorem]{Lemma}
\newtheorem{remark}[theorem]{Remark}
\newtheorem{definition}[theorem]{Definition}

\newcommand{\defnstyle}{\emph}
%\newcommand{Venumerate}{enumerate}
%\newenvironment{Venumerate}{\begin{enumerate}}{\end{enumerate}}
\newcommand{\rec}[1]{\frac{1}{#1}}


\newif\ifsolns
\newif\ifmathlab


\newcommand{\soln}{\textbf{ANSWER: }}

\newenvironment{alphenumerate}%
   {\begin{enumerate}%
      \renewcommand{\theenumi}{\alph{enumi}}%
      \renewcommand{\labelenumi}{(\theenumi)}}
   {\end{enumerate}}





\def\textdefn{\textbf{Definition:~}}

% \cleartoevenpage acts like \newpage, but guarantees that you'll be at the start of a two-page spread.
\newcommand*\cleartoevenpage{%
  \clearpage%
  \ifodd\value{page}\thispagestyle{empty}\begin{center}\em This page intentionally left blank.\end{center}\clearpage\fi}

\newcommand*\cleartooddpage{%
  \clearpage%
  \ifodd\value{page} \else\thispagestyle{empty}\begin{center}\em This page intentionally left blank.\end{center}\clearpage\fi}


\newlength\boxedblankwidth
\setlength{\boxedblankwidth}{.98\textwidth}
\advance\boxedblankwidth by -\leftmargin
\newcommand\boxedblank[2][1in]{\fbox{\raisebox{0pt}[\height][#1]{\parbox[t]{\boxedblankwidth}{#2}}}}

\newlength\textwidthinlist
\newcommand\computetextwidthinlist{\textwidthinlist=\textwidth
                                   \advance\textwidthinlist by -\leftmargin
                                   \advance\textwidthinlist by -\rightmargin}

%\HOMEWORK creates the heading for the homework section, including a \newpage command and an entry in the table of contents.
%\HOMEWORK* creates the same heading and TOC entry, but no \newpage command.
%\HOMEWORK** merely typesets the heading, but does not force a \newpage and does not put anything in the table of contents.
%            This is good for a section containing nothing but the homework questions.
\newcounter{hwkno}
\setcounter{hwkno}{1}

\makeatletter
  \newcommand{\HOMEWORK}{\@ifstar\HOMEWORKatleastonestar\HOMEWORKnostar}
  \newcommand{\HOMEWORKnostar}{%
  \cleartooddpage \pagestyle{hwk}
	\ifmathlab \solnsfalse \fi
%  	\cleardoublepage
%    \newpage
    \HOMEWORKheading \stepcounter{hwkno} 
    \HOMEWORKcontentsline}
  \def\HOMEWORKatleastonestar{\HOMEWORKcheckforsecondstar}
  \newcommand{\HOMEWORKcheckforsecondstar}{\@ifstar\HOMEWORKdoublestar\HOMEWORKsinglestar}
  \newcommand{\HOMEWORKsinglestar}{%
    \HOMEWORKheading
    \HOMEWORKcontentsline}
  \newcommand{\HOMEWORKdoublestar}[1]{%
    \HOMEWORKheading}
  \def\HOMEWORKheading{\subsubsection*{Homework \thehwkno}}
  \def\HOMEWORKcontentsline{\addcontentsline{toc}{subsubsection}{Homework}}
\makeatother

\newcommand{\ENDHOMEWORK}{\cleartooddpage  \ifmathlab \solnstrue \fi \pagestyle{normal}}

\def\pageover{%
  \ifodd\value{page}%
    \par\rule{0pt}{0pt}\hfill\emph{(over)}%
  \fi}

\def\solnsnewpage{}
\def\perspectiveHWnewpage{}

% Use \hwnewpage in the homework sections instead of \newpage.
% This guarantees that if you're at the bottom of an odd (right-hand) page,
% the word "(over)" will appear to remind students there's more homework on the next page.
\def\hwnewpage{%
  \pageover
  \clearpage
}







\newcounter{currentyear}
\setcounter{currentyear}{\theyear}
\newcounter{tempyear}
\newcounter{tempyear2}


\makeatletter
\let\xa=\expandafter
\newcommand\declareproblemlettering[1]{%
  \newcounter{#1@enum}%
  \setcounter{#1@enum}{0}%
  \newenvironment{#1enumerate}%
    {\begin{list}{\stepcounter{#1@enum}%
                  #1\arabic{#1@enum}.%
                  \xdef\@currentlabel{#1\arabic{#1@enum}}%
                  \gdef\p@enumii{#1\arabic{#1@enum}}%
                  %\gdef\labelenumii{(\theenumii)}%
                  \gdef\theenumii{(\alph{enumii})}%
                  \gdef\labelenumii{\theenumii}%
                  }{}
     \@enumdepth=1}%
    {\end{list}}%
}

\newcounter{numstudentsolns}
\setcounter{numstudentsolns}{0}

\newcommand\studentsoln[1]{%
  %Saves the parameter as a student solution
  \stepcounter{numstudentsolns}%
  %\showthe\c@numstudentsolns
  \xa\xdef\csname studentsoln-label-\roman{numstudentsolns}\endcsname{\@currentlabel}%
  %\xa\show\csname studentsoln-label-\roman{numstudentsolns}\endcsname%
  \xa\long\xa\gdef\csname studentsoln-\roman{numstudentsolns}\endcsname{#1}%
  %\xa\show\csname studentsoln-\roman{numstudentsolns}\endcsname%
}

\newcommand\saveasstudentsoln[1]{%
  \studentsoln{#1}%
  #1%
}

\newcommand\gradersoln[1]{%
  %Gives text that shows up for graders' manual only.
  \ifsolns
    #1%
  \fi
}



\newcommand\solution{%
  \@ifnextchar*\soln@star\soln@nostar%
}
\long\def\soln@star*#1{%
  \studentsoln{#1}%
  \gradersoln{\par\soln #1}%
}
\long\def\soln@nostar#1{%
  \gradersoln{\par\soln #1}%
}

\newcommand\esolution{%
  \@ifnextchar*\esoln@star\esoln@nostar%
}
\long\def\esoln@star*#1{%
  \edef\temp@soln{#1}%
  \xa\soln@star\xa*\xa{\temp@soln}%
}
\long\def\esoln@nostar#1{%
  \edef\temp@soln{#1}%
  \xa\soln@nostar\xa{\temp@soln}%
}

\newif\ifstudentsolns
\newif\ifgradersolns

\newcommand\studentsolnscale[1][0.5]{%
  \ifstudentsolns
    \tikzset{scale=#1}
  \fi
}
\newcommand\sspar{%
  \ifstudentsolns
    ~\par
  \fi
}



\newcounter{numattributions}
\setcounter{numattributions}{0}

\newcommand\saveattribution[1]{%
  %Saves the parameter to be displayed in the attributions section
  \stepcounter{numattributions}%
  \edef\attributionlabelname{attribution@page@\roman{numattributions}}
  \xa\label\xa{\attributionlabelname}
  \xa\long\xa\gdef\csname attribution-\roman{numattributions}\endcsname{#1}%
}


\makeatother


%--------------------------------------------------------------------
\makeatletter
%                            \fillwithlines


% \fillwithlines takes one argument, which is either a length or \fill
% or \stretch{number}, and it fills that much vertical space with
% horizontal lines that run the length of the current line.  That is,
% they extend from the current left margin (which depends on whether
% we're in a question, part, subpart, or subsubpart) to the right
% margin.
%
% The distance between the lines is \linefillheight, whose default value
% is set with the command
%
% \setlength\linefillheight{.25in}
%
% This value can be changed by giving a new \setlength command.
%
% The thickness of the lines is \linefillthickness, whose default value
% is set with the command
%
% \setlength\linefillthickness{.1pt}
%
% This value can be changed by giving a new \setlength command.


\newlength\linefillheight
\newlength\linefillthickness
\setlength\linefillheight{.25in}
\setlength\linefillthickness{0.1pt}

\newcommand\linefill{\leavevmode
    \leaders\hrule height \linefillthickness \hfill\kern\z@}


\def\fillwithlines#1{%
  \begingroup
  \ifhmode
    \par
  \fi
  \hrule height \z@
  \nobreak
  \setbox0=\hbox to \hsize{\hskip \@totalleftmargin
          \vrule height \linefillheight depth \z@ width \z@
          \linefill}%
  % We use \cleaders (rather than \leaders) so that a given
  % vertical space will always produce the same number of lines
  % no matter where on the page it happens to start:
  \cleaders \copy0 \vskip #1 \hbox{}%
  \endgroup
}
\makeatother
%--------------------------------------------------------------------

\newcounter{lettercount}
\newcounter{numbercount}
\setcounter{lettercount}{1}
\setcounter{numbercount}{1}

\newcommand{\vletter}{\Alph{lettercount}. \stepcounter{lettercount}&}
\newcommand{\vnumber}{\arabic{numbercount}. \stepcounter{numbercount}&}

%-------------------For Drawings-------------------------------------------------
\usepackage{tikz}
\usepackage{pgfplots}
\pgfplotsset{compat=1.12}

% ------------------------For Putting Student Names in Document ----------------------
%\usepackage{pgfplots}
\usepackage{pgf}
%\pgfplotsset{compat=1.11}


%\pgfmathsetseed{\theyear} % to ensure that the solutions and the original have the same names.
\pgfmathsetseed{\number\pdfrandomseed} % to ensure that it is randomized
\def\N{\thestudentnum} % number of elements in list
\newboolean{unique}
\newcounter{A}
\newcounter{B}

\newcommand{\studentrand}{
	\pgfmathsetmacro\tmplist{random(1,\N)} % first random item
	\readlist*\mylist\tmplist % convert to array
	\whiledo{\mylistlen<\N}{ % try until list is of length \N
		\pgfmathsetmacro\r{random(1,\N)} % generate new random item
		\setboolean{unique}{true}
		\foreachitem \i \in \mylist{ % loop over previous elements
			\ifnum \i=\r % look if new number in list
				 \setboolean{unique}{false} % already in list
				 \breakforeach
			\fi
		}
		\ifthenelse{\boolean{unique}}{ % found unique item !
			\xdef\tmplist{\tmplist,\r} % add new unique item
			\readlist*\mylist\tmplist % convert to array
		}{}
}}

\newcommand{\randstudent}[1][1]{\setcounter{A}{\mylist[#1]}\ignorespaces\getdata[\theA]\studentfirst{}%
}%
\newcommand{\randlast}[1][1]{\setcounter{A}{\mylist[#1]}\ignorespaces\getdata[\theA]\studentlast{}}%
\newcommand{\randpronoun}[1][1]{\setcounter{A}{\mylist[#1]}\ignorespaces\getdata[\theA]\heshe{}}
\newcommand{\randPronoun}[1][1]{\setcounter{A}{\mylist[#1]}\ignorespaces\getdata[\theA]\Heshe{}}
\newcommand{\randpronouns}[1][1]{\setcounter{A}{\mylist[#1]}\ignorespaces\getdata[\theA]\hisher{}}
\newcommand{\randPronouns}[1][1]{\setcounter{A}{\mylist[#1]}\ignorespaces\getdata[\theA]\Hisher{}}

