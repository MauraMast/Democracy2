\section{Weighted Voting Systems}

\begin{enumerate}
  \item Notes on weighted voting systems: \index{weighted voting}
	
	\begin{itemize}
		\item $P_i$ \vfill
		\item $w_i$ \vfill
		\item $q$\vfill
		\item $[q|w_1,w_2,\dots,w_n]$ \vfill
	\end{itemize}

\ifsolns \par\soln
	Weighted voting systems are when there are $n$ voters labeled $P_1\dots P_n$, and each voter casts a vote with a certain weight, $w_i$.  Think of this as having each voter being able to cast $w_i$ votes all for the same decision.  In order for a motion to pass, the sum of all the votes needs to be at least $q$, the quota.  We need the quota to be more than 50\% of the sum of all the votes and no more than the sum of all the votes, but it can be anything in between.  We give a weighted voting system as 
	$[q|w_1,w_2,\dots,w_n]$.
\else          \vfill \fillwithlines{\stretch{1}}\fi
  \clearpage
  \item In a weighted voting system with weights $[30, 29, 16, 8, 3, 1]$,
        if a two-thirds majority of votes is needed to pass a motion, what is the quota? \index{quota}
        \ifsolns
					\fbox{58}
				\else
					          \fillwithlines{\stretch{1}}
				\fi

  \item Consider the weighted voting system $[14, 9, 8 ,5]$.
		\begin{enumerate}
          \item What is the largest reasonable quota for this system?
                  \ifsolns
					\fbox{36}
				\else
					          \fillwithlines{\stretch{1}}
				\fi
          \item What is the smallest reasonable quota for this system?
                  \ifsolns
					\fbox{19}
				\else
					          \fillwithlines{\stretch{1}}
				\fi
		\end{enumerate}
  \item Consider the weighted voting system $[20|7,5,4,4,2,2,2,1,1]$.
        \begin{enumerate}
          \item How many voters are there?
        \ifsolns
					\fbox{9}
				\else
					          \fillwithlines{\stretch{1}}
				\fi
          \item What is the quota?
        \ifsolns
					\fbox{20}
				\else
					          \fillwithlines{\stretch{1}}
				\fi
          \item What is the weight for voter $P_2$?
         \ifsolns
					\fbox{5}
				\else
					          \fillwithlines{\stretch{1}}
				\fi
         \item If the first 4 voters vote for a motion and the rest vote against, does the motion pass?
                \ifsolns
					\fbox{Yes}
				\else
					          \fillwithlines{\stretch{1}}
				\fi
        
          \item If $P_1$ and $P_2$ vote against a motion, will the motion pass?
        \ifsolns
					\fbox{No}
				\else
					          \fillwithlines{\stretch{1}}
				\fi
        \end{enumerate}

\clearpage

  \item What is peculiar about each of the following weighted voting systems?
	
        \begin{enumerate}
        \item $[20|10, 10, 9]$
					\ifsolns
						\fbox{Cannot win without $P_1$ and $P_2$.}
					\else
						\vfill          \fillwithlines{\stretch{1}}
					\fi
				%\end{enumerate}
		
        \item $[7|4, 2, 1]$
        \ifsolns
					\fbox{Everyone must vote for the motion to pass}
				\else
					\vfill          \fillwithlines{\stretch{1}}
				\fi
        %\end{enumerate}

		\item $[51|50, 49, 1]$
        \ifsolns
					\fbox{First voter must vote for the motion for the motion to pass.}
				\else
					\vfill          \fillwithlines{\stretch{1}}
				\fi
        %\end{enumerate}
		
        \item $[6|6, 2, 1, 1]$
        \ifsolns
					\fbox{First voter is the only one who makes a difference.}
				\else
					\vfill          \fillwithlines{\stretch{1}}
				\fi
        %\end{enumerate}
		
		\end{enumerate}
		
\pagebreak
    \item A \defnstyle{dummy} is \ldots 
		    \ifsolns
					\par\soln A voter whose vote does not make any difference on whether a motion passes or not.
				\fi
        %\end{enumerate}
          \fillwithlines{\stretch{1}} \index{dummy}
	
    \item A \defnstyle{dictator} is \ldots 
		    \ifsolns
					\par\soln A voter where their vote is the only reason a motion passes.
				\fi
		          \fillwithlines{\stretch{1}} \index{dictator}
    \item A voter has \defnstyle{veto power} if \ldots 
		    \ifsolns
					\par\soln A voter whose must vote for a motion in order for it to pass.
				\fi
		          \fillwithlines{\stretch{1}} \index{veto power}
    \item In the weighted voting system $[12|9,5,4,2]$, are there any dummies or dictators?
		
		\ifsolns 2 is a dummy. \vfill \else \vspace{1.5in} \fi

    \item In designing a weighted voting system $[q|6,5,4,3,2,1]$, what is the largest quota $q$
          you could pick without giving veto power to anyone?
					
					\ifsolns 15. You need the rest of the voters to be able to pass a motion. \vfill \else \vspace{1.5in} \fi
    
    \item In the weighted voting system $[q|8,5,4,1]$, if every voter has veto power, what is the quota $q$?
		
		\ifsolns 18. The last voter needs to have veto power, so the only way a motion passes is if the vote is unanimous.\vfill \else \vspace{1.5in} \fi
                    

\clearpage
\item A committee has four members ($P_1, P_2, P_3,$ and $P_4$).  In this committee, $P_1$ has twice as many votes as $P_2$; $P_2$ has twice as many votes as $P_3$: $P_3$ has twice as many votes as $P_4$.  Describe the committee as a weighted voting system when the requirements to pass a motion are
\begin{enumerate}
	\item at least two-thirds of the votes 
		\ifsolns \fbox{[10|8,4,2,1]} \else \vfill           \fillwithlines{\stretch{1}} \fi
	\item more than two-thirds of the votes 
		\ifsolns \fbox{[11|8,4,2,1]} \else \vfill           \fillwithlines{\stretch{1}}\fi
	\item at least 80\% of the votes 
		\ifsolns \fbox{[12|8,4,2,1]} \else \vfill           \fillwithlines{\stretch{1}}\fi
	\item more than 80\% of the votes
		\ifsolns \fbox{[13|8,4,2,1]} \else \vfill           \fillwithlines{\stretch{1}}\fi
\end{enumerate}
\end{enumerate}

%</WORKSHEETS>

%<*HWHEADER>
\HOMEWORK
%</HWHEADER>

%<*HOMEWORK>

\begin{Venumerate}

  \item Alice, Bob, Charles, and Danielle are the stockholders in Alphabet Industries, Inc.
        Alice owns 252 shares, Bob owns 741 shares, Charles inherited 637 shares,
        and 412 shares are in Danielle's hands.
        As usual, each share corresponds to a vote in the stockholder's meeting.
        \begin{enumerate}
          \item If a certain type of motion requires a majority vote,
                what is the smallest number of votes needed to pass the motion?
                \solution*{%
                  \fbox{1022.}  
                  \ifgradersolns
                    (Partial credit for 1021.)
                  \fi
                }\vfill           \fillwithlines{\stretch{1}}
                
          \item A different type of motion requires a $2/3$ vote to pass.
                What is the smallest number of votes needed to pass this motion?
                \solution{\fbox{1362.}} \vfill           \fillwithlines{\stretch{1}}
          \item Using the quota you found in part (b), express the weighted voting system
                in the correct notation (with brackets and quota).
                \solution*{\fbox{$[1362 | 741, 637, 412, 252]$.}}\vfill           \fillwithlines{\stretch{1}}
        \end{enumerate}
  \item Which voters have veto power in the system $[51 | 29, 21, 8, 3, 1]$?
                \ifsolns
                  \par\soln \fbox{$P_1$ and $P_2$} (the 29 and the 21) have veto power.
                \fi             \vfill%\fillwithlines{\stretch{1}}


\hwnewpage
  \item Find all dictators, dummies, and voters with veto power in the following weighted voting systems:
        \begin{enumerate}
          \item $[51 | 20, 20, 20]$
                \solution{No dictators, no dummies, all three have veto power.}   \vfill       %\fillwithlines{\stretch{1}}
          \item $[51 | 36, 34, 23, 6]$
                \solution*{No dictators, $P_4$ (the 6) is a dummy, no one has veto power.}        \vfill     %\fillwithlines{\stretch{1}}
          \item $[25 | 27, 11, 7, 2]$
                \solution*{$P_1$ (the 27) is a dictator (and so has veto power); the rest are dummies.}   \vfill%          \fillwithlines{\stretch{1}}
          \item $[31 | 15, 13, 6, 4, 2]$
                \solution{No dictators;
                        $P_1$ and $P_2$ (the 15 and the 13) have veto power;
                        $P_5$ (the 2) is a dummy.}            \vfill% \fillwithlines{\stretch{1}}
        \end{enumerate}

%\hwnewpage
	\item In 1958, the Treaty of Rome established the European Economic Community (EEC) and instituted a
          weighted voting system for the EEC's governance.  
          The members at that time were France, Germany, Italy, Belgium, the Netherlands, and Luxembourg.  
          The three largest countries (France, Germany and Italy) were each given a vote with
          weight 4, Belgium and the Netherlands had votes of weight 2 and Luxembourg's vote had weight 1.  
          The quota was 12.
		
          What is unusual or interesting about this weighted voting system?
          \ifsolns
            \par\soln \fbox{Luxembourg is a dummy.}
          \fi
	      \vfill           \fillwithlines{\stretch{2}}

\end{Venumerate}

\ENDHOMEWORK %</HOMEWORK>

%<*WORKSHEETS>

\cleartooddpage
\section{Banzhaf Power Index} \index{power index!Banzhaf}

\begin{enumerate}

    \item A \defnstyle{coalition} is \ldots \index{coalition}
		\ifsolns
			any set of players that might join forces and vote the same way.  In principle, we can have a coalition with as few as \emph{one} player and as many as \emph{all} players.  The coalition consisting of all the players is called the \textbf{grand coalition}. Since coalitions are just sets of players, the most convenient way to describe coalitions mathematically is to use the \emph{set} notation.
		\else
			\vspace{1in}
		\fi
    \item \begin{enumerate}
            \item Consider a weighted voting system with three voters $P_1$, $P_2$, and $P_3$.
                  List all the coalitions.
                  How many are there?
									
									\ifsolns 
									\(  \begin{array}{c}
										\{P_1\}, \{P_2\}, \{P_3\}\\
										\{P_1, P_2\}, \{P_1, P_3\}, \{P_2, P_3\}\\
										\{P_1, P_2, P_3\}
									\end{array} \)
									\fi
                  \vfill
            \item Consider a weighted voting system with four voters $P_1$, $P_2$, $P_3$, and $P_4$.
                  List all the coalitions.
                  How many are there?
									
									\ifsolns 
									\(  \begin{array}{c}
										\{P_1\}, \{P_2\}, \{P_3\}, \{P_4\}\\
										\{P_1, P_2\}, \{P_1, P_3\}, \{P_1, P_4\}, \{P_2, P_3\}, \{P_2, P_4\}, \{P_3, P_4\}\\
										\{P_1, P_2, P_3\}, \{P_1, P_3, P_4\}, \{P_1, P_2, P_4\}, \{P_2, P_3, P_4\}\\
										\{P_1, P_2, P_3, P_4\}
									\end{array} \)
									\fi
                  \vfill
            \item If a weighted voting system has $n$ voters $P_1$, $P_2$, \ldots, $P_n$,
                  how many coalitions are there?
    \ifsolns
				\fbox{$2^n-1$} \vfill
		\else
			\fillwithlines{\stretch{1}}
		\fi
          \end{enumerate}
\clearpage
    \item A \defnstyle{winning coalition} is \ldots \index{coalition!winning}
				\ifsolns
			one that has enough votes to win.  A single player coalition can be a winning coalition only when that player is a dictator, so under the assumption that there are no dictators in our weighted voting system (dictators are boring) a winning coalition must have at least two players.
		\else
			\fillwithlines{\stretch{1}}
		\fi
    \item List all the winning coalitions in the weighted voting system $[10 | 5, 4, 3, 2, 1]$.
		
		\ifsolns \(\begin{array}{c}
			\{P_1, P_2, P_3\}, \{P_1, P_2, P_4\},\{P_1, P_2, P_5\},  \{P_1, P_3, P_4\}\\
			\{P_1, P_2, P_3, P_4\}, \{P_1, P_2, P_3, P_5\}, \{P_1, P_2, P_4, P_5\}, \{P_2, P_3, P_4, P_5\}\\
			\{P_1, P_2, P_3, P_4, P_5\}
		\end{array}\)\fi
          \vfill
    \item In the weighted voting system \[ [10 | 6, 4, 3, 2, 1],\] 
          consider \textbf{the winning coalition} $\{P_1, P_2, P_3, P_4\}$.\\
          Which voter(s) could change their minds and vote ``no'' without changing the outcome of the vote?
          Which voter(s) \emph{need} to keep voting ``yes'' in order for the motion to pass?
					
					\ifsolns \( P_3, P_4\) could change their minds without changing the outcome. \(P_1\) and \(P_2\) will change the outcome if they change their minds.\fi
          \vfill
    \item A \defnstyle{critical voter} in a winning coalition is \ldots \index{critical voter}
			\ifsolns
			the coalition must have that player's votes to win. \vfill
		\else
			\fillwithlines{\stretch{1}}
		\fi

\clearpage
%\newcommand*\circled[1]{\raisebox{.5pt}{\textcircled{\raisebox{-.9pt} {#1}}}}

		
   \item Consider the voting system $[19|11,9,8,5]$.   % $[21|10,8,5,3,2]$.
         \begin{enumerate}
           \item List all the \emph{winning} coalitions.
					
					\ifsolns{
					\centering
					\tikz[inner sep=.25ex,baseline=-.75ex] \node[circle,draw] {$P_1$}; \tikz[inner sep=.25ex,baseline=-.75ex] \node[circle,draw] {$P_2$}; \ \tikz[inner sep=.25ex,baseline=-.75ex] \node[circle,draw] {$P_1$};\tikz[inner sep=.25ex,baseline=-.75ex] \node[circle,draw] {$P_3$};, \\
					\tikz[inner sep=.25ex,baseline=-.75ex] \node[circle,draw] {$P_1$};\tikz[inner sep=.25ex,baseline=-.75ex] \node {$P_2 P_3$};,\ 
					\tikz[inner sep=.25ex,baseline=-.75ex] \node[circle,draw] {$P_1$};\tikz[inner sep=.25ex,baseline=-.75ex] \node[circle, draw] {$P_2$}; \tikz[inner sep=.25ex, baseline=-.75ex] \node {$P_4$};, \ 
					\tikz[inner sep=.25ex,baseline=-.75ex] \node[circle,draw] {$P_1$};\tikz[inner sep=.25ex,baseline=-.75ex] \node[circle, draw] {$P_3$}; \tikz[inner sep=.25ex, baseline=-.75ex] \node {$P_4$};, \ 
					\tikz[inner sep=.25ex,baseline=-.75ex] \node[circle,draw] {$P_2$};\tikz[inner sep=.25ex,baseline=-.75ex] \node[circle, draw] {$P_3$}; \tikz[inner sep=.25ex, baseline=-.75ex] \node[circle, draw] {$P_4$};, \ 
				\\
					$P_1 P_2 P_3 P_4$,
					
					} \else \vfill \fi
                 
           \item In each winning coalition above, circle the \defnstyle{critical voters}. \ifsolns \vfill \fi
           \item Count the number of times each voter is a critical voter.
                 This is called that voter's \defnstyle{Banzhaf power}. \index{power index!Banzhaf}
                 \begin{center}
                   \begin{tabular}{l|c}
                     Voter & Banzhaf power \\ \hline
										\ifsolns
										 $P_1$ & 5 \\ \hline
                     $P_2$ & 3 \\ \hline
                     $P_3$ & 3 \\ \hline
                     $P_4$ & 1 \\ \hline
                   \else
                     $P_1$ & \raisebox{0pt}[15pt][5pt]{} \\ \hline
                     $P_2$ & \raisebox{0pt}[15pt][5pt]{} \\ \hline
                     $P_3$ & \raisebox{0pt}[15pt][5pt]{} \\ \hline
                     $P_4$ & \raisebox{0pt}[15pt][5pt]{} \\ \hline
										\fi
                   \end{tabular}
                 \end{center}
           \item Add up all the voters' Banzhaf powers; this sum is called the \defnstyle{total Banzhaf power} of the voting system. \ifsolns \textbf{8} \fi
                 \vspace{0.5in}
           \item Finally, divide each voter's Banzhaf power by the total Banzhaf power.
                 The percentage that results is called the voter's \defnstyle{Banzhaf power \underline{index}}.
                 \begin{center}
                   \begin{tabular}{l|c}
                     Voter & Banzhaf power \emph{index}\\ \hline
										\ifsolns
										 $P_1$ & 5/8 \\ \hline
                     $P_2$ & 3/8 \\ \hline
                     $P_3$ & 3/8 \\ \hline
                     $P_4$ & 1/8 \\ \hline
										\else
                     $P_1$ & \raisebox{0pt}[15pt][5pt]{} \\ \hline
                     $P_2$ & \raisebox{0pt}[15pt][5pt]{} \\ \hline
                     $P_3$ & \raisebox{0pt}[15pt][5pt]{} \\ \hline
                     $P_4$ & \raisebox{0pt}[15pt][5pt]{} \\ \hline \fi
                   \end{tabular}
                 \end{center}
         \end{enumerate}

\clearpage
	\item Calculate the Banzhaf Power Index for each voter in the weighted voting system \[[51| 32, 22, 12]\].
	
	\ifsolns The first two voters have a power of 50\% and the last voter has no power. \fi
          \vfill
    \item Make up a weighted voting system with a dummy, and calculate the Banzhaf Power Index for the dummy.
		
		\ifsolns
			Dummies never have any power.
		\fi
          \vfill
    \item Make up a weighted voting system with a dictator, and calculate the Banzhaf Power Index for the dictator.
				
		\ifsolns
			Dictators always have all the power.
		\fi
          \vfill
    \item Make up a weighted voting system in which several voters have veto power.
          Calculate the Banzhaf Power Index for the voters with veto power.  What do you notice?
          \vfill
							
		\ifsolns
			All the voters with veto power have the same amount of power and they have more power than any voter without veto power. \vfill
		\else
			\fillwithlines{\stretch{1}}
		\fi
\clearpage
	\item The U.N. Security Council consists of 15 member countries--5 permanent members and 10 non-permanent members.  A motion can pass only if it has the vote of \emph{all five} of the permanent members plus at least four of the non-permanent members.
	
	\ifsolns \textbf{This problem is more advanced than most of the students are comfortable with, but is excellent for your more mathematically comfortable students.} \fi
	\begin{enumerate}
		\item Describe the critical players in a winning coalition.  \label{UNSC1}
		\fillwithlines{\stretch{1}}
		\item Use your answer in (a),%\ref{UNSC1}, 
		together with the fact that there are 210 nine-member winning coalitions and 638 winning coalitions with 10 or more members, to explain why the total number of times all players are critical is 5080. \label{UNSC2}\fillwithlines{\stretch{1}}
		\item Using the results of (a) and (b), %\ref{UNSC1} and \ref{UNSC2}, 
		show that the Banzhaf power index of a permanent member is given by the ratio $848/5080$. \label{UNSC3} \vfill
		\item Using the results of (a), (b) and (c), %\ref{UNSC1}, \ref{UNSC2} and \ref{UNSC3}, 
		show that the Banzhaf power index of a non-permanent member is given by the ratio $84/5080$. \vfill
		\item Explain why the U.N. Security Council is equivalent to the weighed voting system in which each non-permanent member has 1 vote, each permanent member has 7 votes and the quota is 39 votes. \fillwithlines{\stretch{1}}
	\end{enumerate}
\end{enumerate}

%</WORKSHEETS>

%<*HWHEADER>
\HOMEWORK
%</HWHEADER>

%<*HOMEWORK>

\begin{Venumerate}
  \item List all the winning coalitions in the weighted voting system $[12|7,5,4,2]$.
        \solution*{
          For convenience, we list them both by $P$-number and by their weights:\par
          \[\boxed{\begin{array}{ll}
              \{P_1, P_2, P_3, P_4\} & \{7,5,4,2\} \\
              \{P_1, P_2, P_3\} & \{7,5,4\} \\
              \{P_1, P_2, P_4\} & \{7,5,2\} \\
              \{P_1, P_3, P_4\} & \{7,4,2\} \\
              \{P_1, P_2\} & \{7,5\} \\
            \end{array}}\] }
						
					\vfill\vfill
  \item List all the winning coalitions in the weighted voting system $[11|6,4,3,3,1]$.
        \solution{
          For convenience, we list them both by $P$-number and by their weights:\par
      \[\boxed{\begin{array}{ll}
              \{P_1, P_2, P_3, P_4, P_5\} & \{6,4,3,3,1\} \\
              \{P_1, P_2, P_3, P_4\} & \{6,4,3,3\} \\
              \{P_1, P_2, P_3, P_5\} & \{6,4,3,1\} \\
              \{P_1, P_2, P_4, P_5\} & \{6,4,3,1\} \\
              \{P_1, P_3, P_4, P_5\} & \{6,3,3,1\} \\
              \{P_2, P_3, P_4, P_5\} & \{4,3,3,1\} \\
              \{P_1, P_2, P_3\} & \{6,4,3\} \\
              \{P_1, P_2, P_4\} & \{6,4,3\} \\
              \{P_1, P_2, P_5\} & \{6,4,1\} \\
              \{P_1, P_3, P_4\} & \{6,3,3\} \\
            \end{array}}\]
        }\vfill\vfill
\vfill  \item In the weighted voting system \[ [38|22,20,17,9,5], \] consider the winning coalition
        $\{P_2, P_3, P_4, P_5\}$.
        Which voters are critical voters in this coalition?
        \solution*{$P_2$ and $P_3$ are critical voters (the 20 and the 17).}\vfill
  \vfill\item In the weighted voting system \[ [7|3,3,2,2,2,1], \] consider the winning coalition
        $\{P_1,P_3,P_4,P_6\}$.
        Which voters are critical voters in this coalition?
        \solution{$P_1$, $P_3$, and $P_4$ are critical voters (the 3, 2, and 2).}\vfill
  \vfill\item Calculate the Banzhaf Power Index for each voter in the weighted voting system \[[34 | 12, 10, 7, 6]\].
        %\ifsolns
          \solution*{All 4 voters have equal Banzhaf power: \fbox{$0.25, 0.25, 0.25, 0.25$.}}\vfill
        %\fi
				  \vfill
					
									
\hwnewpage

\item Calculate the Banzhaf Power Index for each voter in the weighted voting system $[27 | 15, 7, 5]$.
        \ifsolns
          \par\soln \fbox{0.33, 0.33, 0.33}
        \fi
  \vfill

\item Consider the voting system $[25|24,20,1]$.
        \begin{enumerate}
          \item Calculate the percentage of the total weight that each voter holds.
                \solution*{53.3\%, \quad 44.4\%, \quad 2.2\%}
          \vfill\item Calculate the Banzhaf Power Index for each voter.
                \solution*{0.60, \quad 0.20, \quad 0.20} \vfill
          \vfill\item Comparing your answers to parts (a) and (b), explain in complete sentences
                why the weight controlled by the voter is not the same thing as the power held by each voter.
                \ifsolns
                  \par\soln 
                  There are several ways to answer this question.
                  For example, voter $P_2$ has twenty times more weight than voter $P_3$,
                  but they have exactly the same power.
                  Another way to see it is that voter $P_1$ has fully three times as much power as voter $P_2$,
                  even though the weights of 24 and 20 are not that different.
                \fi\vfill
        \end{enumerate}
				
				\hwnewpage
\item Calculate the Banzhaf Power Index for each voter in the weighted voting system \[26 | 15, 13, 7]\].
        \solution*{0.5, \quad 0.5, \quad 0}
  \vfill
	
	
\item Calculate the Banzhaf Power Index for each voter in the weighted voting system \[[63|43,35,22,16]\].
        \solution{0.417,\quad 0.25\quad 0.25\quad 0.0833}
				\vfill
				\hwnewpage

   \item Nassau County, New York used to be governed by a Board of Supervisors.
         The county had six districts, each of which one delegate to vote on county issues.
         The delegates' votes were weighted proportionately to the districts' population in 1964:
         \begin{center}
           \begin{tabular}{l|r}
             District & Weight \\ \hline
             Hempstead \#1 & 31 \\
             Hempstead \#2 & 31 \\
             Oyster Bay    & 28 \\
             North Hempstead & 21 \\
             Long Beach    &  2 \\
             Glen Cove     &  2
           \end{tabular}
         \end{center}
         A simple majority was needed to pass a motion.
         \begin{enumerate}
           \item Express this weighted voting system in our usual notation.
                 \ifsolns
                   \par\soln \fbox{$[ 58 | 31, 31, 28, 21, 2, 2]$}
                 \fi \vfill
           \item Calculate the Banzhaf power of each district.
                 \ifsolns
                   \par\soln A motion will pass if and only if two of the three largest voters vote for it.  Thus the Banzhaf power is: \\
                   \begin{tabular}[c]{l|rr}
                     District        & Weight & Banzhaf Power\\ \hline
                     Hempstead \#1   & 31     & 0.33 \\
                     Hempstead \#2   & 31     & 0.33 \\
                     Oyster Bay      & 28     & 0.33 \\
                     North Hempstead & 21     & 0.00 \\
                     Long Beach      &  2     & 0.00 \\
                     Glen Cove       &  2     & 0.00 \\
                   \end{tabular}
                 \fi \vfill
           \item What percentage of the county population lived in districts that are dummies?
                 \ifsolns
                   \par\soln Since the weights are proportionate to the population, and the total weight is 115,
                   we conclude that $\dfrac{21+2+2}{31+31+28+21+2+2} = \dfrac{25}{115} = \boxed{21.7\%}$
                   of the population had no say at all in county government.
                 \fi\vfill
           \item In 1965 John F.~Banzhaf~III argued in court that even though the weights were proportionate to population,
                 this system of government was unfair.
                 He won!
                 \ifsolns
                   \par\fbox{\emph{(No answer required.)}}
                 \fi
         \end{enumerate}
\vfill
\end{Venumerate}

\ENDHOMEWORK

\section{Shapley-Shubik Power Index} \index{power index!Shapley-Shubik}

\begin{enumerate}

    \item A \defnstyle{sequential coalition} is \ldots  \index{coalition!sequential}
			\ifsolns
				A coalition of all voters in a particular order.  The assumption is that coalitions are formed sequentially: Players join the coalition and cast their votes in an orderly sequence.
			\else
				\fillwithlines{\stretch{1}}
			\fi
    \item \begin{enumerate}
            \item Consider a weighted voting system with three voters $P_1$, $P_2$, and $P_3$.
                  List all the sequential coalitions.
                  How many are there?
									\ifsolns \par $\{P_1,P_2,P_3\}, \{P_1,P_3,P_2\},\{P_2,P_1,P_3\},\{P_3,P_2,P_1\},\{P_2,P_3,P_1\},\{P_3,P_1,P_2\}$,\fbox{6}\fi
                  \vfill
            \item Consider a weighted voting system with four voters $P_1$, $P_2$, $P_3$, and $P_4$.
                  List all the sequential coalitions.
                  How many are there? \ifsolns \fbox{24}\fi
                  \vfill
									\vfill
            \item If a weighted voting system has $n$ voters $P_1$, $P_2$, \ldots, $P_n$,
                  how many sequential coalitions are there? \ifsolns $n!$ \fi \medskip
                  
          \end{enumerate}
\clearpage
    \item A \defnstyle{pivotal player} is \ldots  \index{pivotal player}
			\ifsolns the player who contributes the votes to turn a losing coalition into a winning coalition. \fi
			\fillwithlines{\stretch{1}}
			
    \item List all the sequential coalitions in the weighted voting system $[4 | 3, 2, 1]$ and determine the pivotal player.
          \vfill
    \item In the weighted voting system $[4 | 3, 2, 1]$, 
    			Is there a voter who is always pivotal?
    			Is there a voter who is never pivotal?
          \vfill
	\ifsolns
  	 To calculate the power of a voter, count the number of times the voter is pivotal. Then divide by the total number of sequential coalitions. Calculate the power of each voter in the weighted voting system $[4 | 3, 2, 1]$.  Just like in the previous power index, the sum of all the powers must equal one.
		\fi

\clearpage
   \item Consider the voting system $[6|4,3,2,1]$.   % $[21|10,8,5,3,2]$.
         \begin{enumerate}
           \item List all the \emph{sequential} coalitions.
                 \vfill
           \item In each sequential coalition above, circle the \defnstyle{pivotal voters}.
           \item Count the number of times each voter is a pivotal voter.
                 This is called that voter's \defnstyle{Shapley-Shubik power}. \index{power index!Shapley-Shubik}
                 \begin{center}
                   \begin{tabular}{l|l}
                     Voter & Shapley-Shubik power \\ \hline
										\ifsolns
                     $P_1$ & 10 \\ \hline
                     $P_2$ & 6 \\ \hline
                     $P_3$ & 6 \\ \hline
                     $P_4$ & 2 \\ \hline
										\else
                     $P_1$ & \raisebox{0pt}[15pt][5pt]{} \\ \hline
                     $P_2$ & \raisebox{0pt}[15pt][5pt]{} \\ \hline
                     $P_3$ & \raisebox{0pt}[15pt][5pt]{} \\ \hline
                     $P_4$ & \raisebox{0pt}[15pt][5pt]{} \\ \hline
										\fi
                   \end{tabular}
                 \end{center}
           \item Add up all the voters' Shapley-Shubik powers; this sum is called the \defnstyle{total Shapley-Shubik power} of the voting system.
                 \vspace{0.5in}
           \item Finally, divide each voter's Shapley-Shubik power by the total Shapley-Shubik power.
                 The percentage that results is called the voter's \defnstyle{Shapley-Shubik power \underline{index}}.
                 \begin{center}
                   \begin{tabular}{l|l}
                     Voter & Shapley-Shubik power \emph{index}\\ \hline
                     $P_1$ & \raisebox{0pt}[15pt][5pt]{} \\ \hline
                     $P_2$ & \raisebox{0pt}[15pt][5pt]{} \\ \hline
                     $P_3$ & \raisebox{0pt}[15pt][5pt]{} \\ \hline
                     $P_4$ & \raisebox{0pt}[15pt][5pt]{} \\ \hline
                   \end{tabular}
                 \end{center}
         \end{enumerate}

\clearpage
	\item Calculate the Shapley-Shubik Power Index for each voter in the weighted voting system $[51| 32, 22, 12]$.
          \vfill
    \item Make up a weighted voting system with a dummy, and calculate the Shapley-Shubik Power Index for the dummy.
          \vfill
    \item Make up a weighted voting system with a dictator, and calculate the Shapley-Shubik Power Index for the dictator.
          \vfill
    \item Make up a weighted voting system in which several voters have veto power.
          Calculate the Shapley-Shubik Power Index for the voters with veto power.  What do you notice?
          \vfill

\clearpage
	\item In some cities the city Council operates under what is known as the, ``strong -- mayor''. Under this system the city Council can pass a motion under a simple majority, but the mayor has the power to veto the decision. The mayor's veto can then be overruled by a ``super majority''  \index{super majority} of the council members. As an example, consider the city of Ice-n-knock.  In Ice-n-knock, the city Council has four members plus a strong mayor who has a vote as well as the power to veto motion supported by a simple majority of the council members. On the other hand, the mayor cannot veto a motion supported by all four Council members. Thus, a motion can pass if the mayor +2 or more Council members supported or, alternatively, if the mayor is against it at the four council members support it. 
	
	It makes sense that under these rules, the four council members have the same amount of power, but the mayor has more. Compute the Shapley-Shubik Power Index of this weighted voting system to figure out exactly how much more.
	
	\vfill
	
	For purposes of comparison, calculate the Banzhaf power distribution of Ice-n-knock.
\vfill
\end{enumerate}

%</WORKSHEETS>

%<*HWHEADER>
\HOMEWORK
%</HWHEADER>

%<*HOMEWORK>

\begin{Venumerate}
  \item List all the sequential coalitions in the weighted voting system $[16|9,8,7]$.
        %\ifsolns \par
				\solution*{
        For convenience, we list them both by $P$-number and by their weights: 
          \[\boxed{\begin{array}{ll}
               \{P_1, P_2, P_3\} & \{9,8,7\} \\
							 \{P_1, P_3, P_2\} & \{9,7,8\} \\
							 \{P_2, P_1, P_3\} & \{8,9,7\} \\
							 \{P_2, P_3, P_1\} & \{8,7,9\} \\
							 \{P_3, P_1, P_2\} & \{7,9,7\} \\
							 \{P_3, P_2, P_1\} & \{7,8,9\} \\
            \end{array}}\]
						}
        %\fi
  \vfill \item List all the sequential coalitions in the weighted voting system $[51|40,30,20,10]$.
        \solution{
          For convenience, we list them both by $P$-number and by their weights:
          \[\boxed{\begin{array}{llll}
 \{P_1, P_2, P_3, P_4\} & \{P_2, P_1, P_3, P_4\} & \{P_3, P_1, P_2, P_4\} & \{P_4, P_1, P_2, P_3\} \\
\{P_1, P_2, P_4, P_3\} & \{P_2, P_1, P_4, P_3\} & \{P_3, P_1, P_4, P_2\} & \{P_4, P_1, P_3, P_2\} \\
\{P_1, P_3, P_2, P_4\} & \{P_2, P_3, P_1, P_4\} & \{P_3, P_2, P_1, P_4\} & \{P_4, P_2, P_1, P_3\} \\
\{P_1, P_3, P_4, P_2\} & \{P_2, P_3, P_4, P_1\} & \{P_3, P_2, P_4, P_1\} & \{P_4, P_2, P_3, P_1\} \\

            \end{array}}\]
						 \[\boxed{\begin{array}{llll}
\{40, 30, 20, 10\} & \{30, 40, 20, 10\} & \{20, 40, 30, 10\} & \{10, 40, 30, 20\} \\
\{40, 30, 10, 20\} & \{30, 40, 10, 20\} & \{20, 40, 10, 30\} & \{10, 40, 20, 30\} \\
\{40, 20, 30, 10\} & \{30, 20, 40, 10\} & \{20, 30, 40, 10\} & \{10, 30, 40, 20\} \\
\{40, 20, 10, 30\} & \{30, 20, 10, 40\} & \{20, 30, 10, 40\} & \{10, 30, 20, 40\} \\


            \end{array}}\]
						
        } \vfill
%  \item In the weighted voting system $[38|22,20,17,9,5]$, consider the winning coalition
%        $\{P_2, P_3, P_4, P_5\}$.
%        Which voters are critical voters in this coalition?
%        \solution*{$P_2$ and $P_3$ are critical voters (the 20 and the 17).}
%  \item In the weighted voting system $[7|3,3,2,2,2,1]$, consider the winning coalition
%        $\{P_1,P_3,P_4,P_6\}$.
%        Which voters are critical voters in this coalition?
%        \solution{$P_1$, $P_3$, and $P_4$ are critical voters (the 3, 2, and 2).}
%  \item Calculate the Shapley-Shubik Power Index for each voter in the weighted voting system $[34 | 12, 10, 7, 6]$.
%        \ifsolns
%          \solution*{All four voters have equal Shapley-Shubik power: \fbox{$0.25, 0.25, 0.25, 0.25$.}}
%        \fi

	\hwnewpage
  \item Consider the voting system $[25|24,20,1]$.
        \begin{enumerate}
          \item Calculate the percentage of the total weight that each voter holds.
                \solution*{53.3\%, \quad 44.4\%, \quad 2.2\%}
          \vfill \item Calculate the Shapley-Shubik Power Index for each voter.
                \solution*{0.60, \quad 0.20, \quad 0.20}
          \vfill \item Comparing your answers to parts (a) and (b), explain in complete sentences
                why the weight controlled by the voter is not the same thing as the power held by each voter.
                \ifsolns
                  \par\soln 
                  There are several ways to answer this question.
                  For example, voter $P_2$ has twenty times more weight than voter $P_3$,
                  but they have exactly the same power.
                  Another way to see it is that voter $P_1$ has fully three times as much power as voter $P_2$,
                  even though the weights of 24 and 20 are not that different.
                \fi
        \end{enumerate}
  \vfill 
	\item Calculate the Shapley-Shubik Power Index for each voter in the weighted voting system $[15|16,8,4,1]$.
      \[\boxed{\begin{array}{llll}
\{16, 8, 4, 2\} & \{8, 16, 4, 2\} & \{4, 16, 8, 2\} & \{2, 16, 8, 4\} \\
\{16, 8, 2, 4\} & \{8, 16, 2, 4\} & \{4, 16, 2, 8\} & \{2, 16, 4, 8\} \\
\{16, 4, 8, 2\} & \{8, 4, 16, 2\} & \{4, 8, 16, 2\} & \{2, 8, 16, 4\} \\
\{16, 4, 2, 8\} & \{8, 4, 2, 16\} & \{4, 8, 2, 16\} & \{2, 8, 4, 16\} \\
            \end{array}}\] 
			\ifsolns
          \par\soln \fbox{1,0,0}
        \fi
\hwnewpage
  \vfill \item Calculate the Shapley-Shubik Power Index for each voter in the weighted voting system $[24|16,8,4,1]$.
	      \[\boxed{\begin{array}{llll}
\{16, 8, 4, 2\} & \{8, 16, 4, 2\} & \{4, 16, 8, 2\} & \{2, 16, 8, 4\} \\
\{16, 8, 2, 4\} & \{8, 16, 2, 4\} & \{4, 16, 2, 8\} & \{2, 16, 4, 8\} \\
\{16, 4, 8, 2\} & \{8, 4, 16, 2\} & \{4, 8, 16, 2\} & \{2, 8, 16, 4\} \\
\{16, 4, 2, 8\} & \{8, 4, 2, 16\} & \{4, 8, 2, 16\} & \{2, 8, 4, 16\} \\
            \end{array}}\] 
        \solution*{0.5,  0.5,  0,  0}
  \vfill \item Calculate the Shapley-Shubik Power Index for each voter in the weighted voting system $[28|16,8,4,1]$.
	      \[\boxed{\begin{array}{llll}
\{16, 8, 4, 2\} & \{8, 16, 4, 2\} & \{4, 16, 8, 2\} & \{2, 16, 8, 4\} \\
\{16, 8, 2, 4\} & \{8, 16, 2, 4\} & \{4, 16, 2, 8\} & \{2, 16, 4, 8\} \\
\{16, 4, 8, 2\} & \{8, 4, 16, 2\} & \{4, 8, 16, 2\} & \{2, 8, 16, 4\} \\
\{16, 4, 2, 8\} & \{8, 4, 2, 16\} & \{4, 8, 2, 16\} & \{2, 8, 4, 16\} \\
            \end{array}}\] 
					
			\solution*{0.33, 0.33, 0.33, 0}
%        \solution{0.417,\quad 0.25\quad 0.25\quad 0.0833}
%%   \item Nassau County, New York used to be governed by a Board of Supervisors.
%%         The county had six districts, each of which one delegate to vote on county issues.
%%         The delegates' votes were weighted proportionately to the districts' population in 1964:
%%         \begin{center}
%%           \begin{tabular}{l|r}
%%             District & Weight \\ \hline
%%             Hempstead \#1 & 31 \\
%%             Hempstead \#2 & 31 \\
%%             Oyster Bay    & 28 \\
%%             North Hempstead & 21 \\
%%             Long Beach    &  2 \\
%%             Glen Cove     &  2
%%           \end{tabular}
%%         \end{center}
%%         A simple majority was needed to pass a motion.
%%         \begin{enumerate}
%%           \item Express this weighted voting system in our usual notation.
%%                 \ifsolns
%%                   \par\soln \fbox{$[ 58 | 31, 31, 28, 21, 2, 2]$}
%%                 \fi
%%           \item Calculate the Banzhaf power of each district.
%%                 \ifsolns
%%                   \par\soln A motion will pass if and only if two of the three largest voters vote for it.  Thus the Banzhaf power is: \\
%%                   \begin{tabular}[c]{l|rr}
%%                     District        & Weight & Banzhaf Power\\ \hline
%%                     Hempstead \#1   & 31     & 0.33 \\
%%                     Hempstead \#2   & 31     & 0.33 \\
%%                     Oyster Bay      & 28     & 0.33 \\
%%                     North Hempstead & 21     & 0.00 \\
%%                     Long Beach      &  2     & 0.00 \\
%%                     Glen Cove       &  2     & 0.00 \\
%%                   \end{tabular}
%%                 \fi
%%           \item What percentage of the county population lived in districts that are dummies?
%%                 \ifsolns
%%                   \par\soln Since the weights are proportionate to the population, and the total weight is 115,
%%                   we conclude that $\dfrac{21+2+2}{31+31+28+21+2+2} = \dfrac{25}{115} = \boxed{21.7\%}$
%%                   of the population had no say at all in county government.
%%                 \fi
%%           \item In 1965 John F.~Banzhaf~III argued in court that even though the weights were proportionate to population,
%%                 this system of government was unfair.
%%                 He won!
%%                 \ifsolns
%%                   \par\fbox{\emph{(No answer required.)}}
%%                 \fi
%%         \end{enumerate}
\vfill
\end{Venumerate}


\ENDHOMEWORK %</HOMEWORK>

%<*WORKSHEETS>

\cleartooddpage
