{\large Worksheet for calculating Hamilton's Method:}

\begin{enumerate}
	\item What is the total population? \hrulefill
	\item How many seats are you apportioning\footnote{If you aren't given this number, it is the total of the Standard Quotas}?  \hrulefill
	\item Calculate the Standard Divisor (divide your population by the number of seats):  \hrulefill
	\item Calculate all the Standard Quotas (Q.) (divide the population of each state by your Standard Divisor).  The total of your standard quotas should be the same as the number of seats.  If that is not the case, you have rounded off too much.:
	
	\begin{center}
			\begin{tabular}{l|c|c|c|p{36pt}|p{36pt}} \hline
	State	&	Pop. &Q. &	LQ&  Addi\-tional Seats	 	&  	 	Appor\-tionment \\\hline
\raisebox{0pt}[72pt][72pt]{\makebox[36pt]{}}&\makebox[36pt]{}&\makebox[36pt]{}&\makebox[36pt]{}&\makebox[36pt]{}&\makebox[36pt]{}\\ \hline
Totals &&&&&\\
		\end{tabular}
	\end{center}
	\item Write down all the Lower Quotas (LQ) by cutting off the decimal places.  The total of your lower quotas should be less than the number of seats.  The difference between the two is the additional seats you get to give out.
	\item Give out your additional seats by giving one seat to each state ranked by the stuff to the right of the decimal point on the standard quotas.
\end{enumerate} 